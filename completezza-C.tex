\chapter{Completezza di C}

Innanzitutto, definiamo la distanza fra due numeri $z_1$ e $z_2$:
$$d(z_1,z_2)=|z_1-z_2|=sqrt{(x_1-x_2)^2+(y_1-y_2)^2}$$
Sia ora $\{z_n\}_{n=0}^{\infty}$ una successione in $\C$; come per i reali, il campo $\C$ è completo se, qualora tale successione fosse una successione di Cauchy, allora converge.

Ma procediamo per gradi; iniziamo a definire la convergenza:

\begin{definizione}
 Una successione $z_n$ converge a $z$ se $|z_n-z|\to0$ per $n\to\infty$.
\end{definizione} Ora invece richiamiamo la definizione di successione di Cauchy, vista nel campo reale:
\begin{definizione}
$z_n$ è una successione di Cauchy se $\forall \epsilon$  $\exists N_{\epsilon} : \forall n,m$>$N_{\epsilon} \, |z_n-z_m| < \epsilon$
\end{definizione} Come detto prima, se ogni successione di Cauchy converge allora $\C$ è completo; dimostriamo che si verifica questa situazione:
$$\frac{|x_n-x_m|+|y_n-y_m|}{2}\leq |z_n-z_m|=$$
$$=|(x_n-x_m)+i(y_n-y_m)|\leq |x_n-x_m|+|y_n-y_m|$$
Se $z_n$ è una successione di Cauchy, allora i due addendi dell'ultima disequazione sono entrambi $<\epsilon$ $\forall n,m\leq N_{\epsilon}$; ma allora si ha che $\frac{|x_n-x_m|+|x_n-x_m|}{2} \to 0$ e, dato che è una semisomma di cose positive, $x_n\to x$ e $y_n\to y$, cioè $z_n=x_n+iy_n \to z=x+iy$.