\chapter{Le funzioni a variabile complessa}

Sia $f:D\subset \C \to \C$, dove D è un dominio, cioè un insieme aperto e connesso. Ridefiniamo alcune nozioni base:
\begin{definizione}
f si dice continua in $z_0 \in D$ se $\forall \epsilon$>0 $\exists \delta_{\epsilon , z_0} : \, |f(z)-f(z_0)|$<$\epsilon \, \forall z: \, |z-z_0|$<$ \delta_{\epsilon , z_0} $
\end{definizione}
\begin{definizione}
f è derivabile in $z_0$ se esiste finito il
$$\lim_{z \to z_0} {\frac{f(z)-f(z_0)}{z-z_0} =f'(z_0)}$$
La funzione $f(z)$ si dice olomorfa in $z_0$ se esiste $f'(z_0)$, cioè:
 $$\forall \epsilon>0 \, \, \exists \delta_{\epsilon , z_0} : \, \left|\frac{f(z)-f(z_0)}{z-z_0} - f'{z_0}\right|< \epsilon \, \forall z: \, |z-z_0|<\delta_{\epsilon , z_0} $$
\end{definizione}

Introduciamo ora un importante risultato che lega le derivate parziali di una funzione olomorfa, le condizioni di Cauchy-Riemann:
\begin{teorema}(Condizioni di Cauchy-Riemann)\\
Sia f derivabile in $z_0$, dove f e $z_0$ sono rispettivamente della forma $u+iv$ e $x_0+iy_0$. Allora esistono le derivate parziali di $u$ e $v$ -che, in generale, saranno  funzioni di $x$ e $y$- nel punto $(x_0,y_0)$ e valgono:
$$\frac{\partial u}{\partial x}=\frac{\partial v}{\partial y}$$
$$\frac{\partial u}{\partial y}=-\frac{\partial v}{\partial x}$$
\end{teorema}


\begin{proof}
Per ipotesi, esiste il $\lim_{\xi \to 0} \frac{f(z_0+ \xi)-f(z_0)}{\xi} =f'(z_0)$, con $\xi \in \C$. \\Se $\xi=h \in \R$:
$$\frac{u(x_0 +h,y_0)+iv(x_0 +h,y_0)- (u(x_0,y_0)+iv(x_0 ,y_0)) }{h}=$$
$$=\frac{u(x_0 +h,y_0)-u(x_0,y_0)}{h} +i\frac{v(x_0+h,y_0)-v(x_0 ,y_0)}{h} \to f'(z_0) \text{ per } h \to 0$$
Quindi, otteniamo che 
$$\left. f'(z_0)=\frac{\partial u}{\partial x} \right|_{(x_0,y_0)} +i  \left. \frac{\partial v}{\partial x} \right|_{(x_0,y_0)}$$
Se invece $\xi=ih$, abbiamo:
$$\frac{u(x_0,y_0+h)+iv(x_0,y_0+h)-(u(x_0,y_0)+iv(x_0 ,y_0))}{ih}=$$
$$=\frac{u(x_0,y_0+h)-u(x_0,y_0)}{ih}+i\frac{v(x_0,y_0+h)-v(x_0 ,y_0)}{ih} \to f'(z_0) \text{ per } h \to 0$$
Quindi, otteniamo che 
$$f'(z_0)=\left. \frac{1}{i} \frac{\partial u}{\partial y} \right|_{(x_0,y_0)} +\left. \frac{\partial v}{\partial y} \right|_{(x_0,y_0)}=\left. \frac{\partial v}{\partial y} \right|_{(x_0,y_0)}-i \left. \frac{\partial u}{\partial y} \right|_{(x_0,y_0)}$$
Confrontando i due risultati, otteniamo la tesi.

\end{proof}

Se $u$ e $v$ hanno derivate parziali continue in un disco centrato su $z_0$ e valgono le condizioni di Cauchy-Riemann in $z_0$, allora f è derivabile in $z_0$. Inoltre, continuano a valere le regole di derivazione ricavate per funzioni reali a variabile reale.

\subsection{Il significato della derivata}

Nel campo reale, la derivata rappresentava il coefficiente angolare della retta tangente al grafico in un dato punto; che significato ha nel campo complesso?

Consideriamo una curva $\phi :[a,b] \to \C$ passante per $z_0$ e differeziabile con continuità. Supponiamo $\gamma(t_0)=z_0$; $\dot{\gamma}(t_0)$ è il vettore tangente alla curva in $z_0$. Attraverso la funzione $f$ la curva $\gamma$ va a finire in un'altra curva, $f(\gamma(t))$; chi è il vettore tangente a questa nuova curva? Derivando, otteniamo:
$$\left. \frac{d}{dt}f(\gamma(t))\right|_{t_0} =\dot{\gamma}(t_0) f'(\gamma(t_0)) =\dot{\gamma}(t_0) f'(z_0)$$
Quindi la derivata rappresenta un fattore di scala (locale) che modifica la lunghezza del vettore tangente $f'(z_0)$. Vediamo come si comporta la funzione argomento:
$$arg[\dot{\gamma}(t_0) f'(\gamma(t_0))]=arg[\dot{\gamma}(t_0)]+arg[ f'(\gamma(t_0))]$$ cioè, localmente abbiamo una rotazione fissa.

Per capire meglio tale affermazione, immaginiamo di avere due curve incidenti; ognuna di esse avrà un vettore tangente nel punto di intersezione, e tali vettori avranno un angolo fra di essi. Tali vettori, tramite la derivazione, vengono moltiplicati per lo stesso fattore di scala e vengono ruotati dello stesso angolo; quindi l'angolo fra di essi rimane invariato.

Cosa succede se la derivata si annulla? In questi casi, l'angolo fra i vettori tangenti non viene conservato (ad esempio, immagine tramite la funzione $f(z)=z^2$ del quadrato unitario avente vertice nell'origine degli assi).

\section{Altre proprietà}

Abbiamo visto che, se $f(z)=u(x,y)+iv(x,y)$ è una funzione olomorfa, valgono le condizioni di Cauchy-Riemann, cioè $\frac{\partial u}{\partial x}=\frac{\partial  v}{\partial y}$ e $\frac{\partial  u}{\partial y}=-\frac{\partial v}{\partial x}$. Dalle derivate seconde ricaviamo  una proprietà molto utile, cioè che le funzioni $u$ e $v$ sono  armoniche ($\nabla^2 u=0$, $\nabla^2 v=0$); infatti:
$$\frac{\partial^2 u}{\partial x^2}=\frac{\partial^2 v}{\partial y \partial x}=-\frac{\partial^2 u}{\partial y^2} \text{; ma allora si ha che } \frac{\partial^2 u}{\partial x^2}+\frac{\partial^2 u}{\partial y^2}=0 \text{ e, similmente, } \frac{\partial^2 v}{\partial x^2}+\frac{\partial^2 v}{\partial y^2}=0$$
Quindi, attraverso le formule di Cauchy-Riemann, possiamo passare da $u$ a $v$ e viceversa, a meno di una costante (rappresentata dal segno).