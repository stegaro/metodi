\chapter{Spazi di Hilbert e teoria degli operatori}

La sistematizzazione degli spazi di Hilbert è dovuta a Von Neumann.
\\
\\
Sia $\mathscr{H}$ uno spazio lineare su $\C$, cioè uno spazio nel quale sono definite la somma e il prodotto per uno scalare. Esso si dice \textbf{spazio pre-hilbertiano} se è definito  il \textbf{prodotto interno}, cioè un'operazione $\mathscr{H} \times \mathscr{H} \to \C$ : $<x|x> \geq 0$, e $<x|x> =0 \iff x=0$.\\Quando scambiamo due termini nel prodotto interno così definito, prendiamo il complesso coniugato del nuovo prodotto, cioè si ha che $<x|y>=\overline{<y|x>}$.

Il prodotto interno così definito è lineare nel secondo membro e antilineare nel primo membro; infatti:

\begin{itemize}
\item $<x|\lambda y_1 + \mu y_2>=<x|\lambda y_1> + <x|\mu y_2> =\lambda <x|y_1> + \mu <x|y_2>$
\item $<\lambda x_1+\mu x_2|y>=\overline{<y|\lambda x_1 + \mu x_2>}=\overline{\lambda <y|x_1>+ \mu <y|x_2>} = \overline{\lambda} <x_1|y> + \overline{\mu} <x-2|y>$
\end{itemize} Vediamo ora le proprietà del prodotto scalare:

\begin{itemize}
\item $x$ è ortogonale a $y$ ($x \perp y$) se $<x|y>=0$
\item Introduciamo la quantità $\|x\|=\sqrt{<x|x>}$, che dimostreremo essere la norma; risultano ovvie alcune proprietà:
\begin{itemize}
\item $\|x\| \geq 0$
\item $\|x\| =0 \iff x=0$
\item $\| \lambda x\|=|\lambda| \|x\|$
\end{itemize}
\item Prendiamo $x,y$ tali che $x \perp y$; si ha che $\|x+y\|=\|x\|^2+\|y\|^2$. Infatti: \\$\|x+y\|=<x+y|x+y>=<x|x>+<x|y>+<y|x>+<y|y>=\|x\|^2+\|y\|^2$ \\Questa uguaglianza risulta quindi essere il \textbf{teorema di Pitagora}.
\item $u_k$, con $k=1,2, \dots ,N$ è un \textbf{sistema ortonormale} di vettori di $\mathscr{H}$ se $<u_k|u_k>=\delta_{kj}$
\item \textbf{Disuguaglianza di Bessel}: $\|x\|^2= \sum_{k=1} ^N |<u_k|x>|^2$ \\Dimostriamo questo fatto: chiamiamo $x'=\sum_{k=1} ^N <u_k|x>u_k$, e scriviamo $x$ come $(x-x')+x'$. Mostriamo che $<x-x'|x'>=0$; una volta mostrato questo, si ha che: \\$\|x\|^2 =\|x+x'\|^2 + \|x'\|^2 \geq \|x'\|^2 =<x'|x'>= \sum_{k=1} ^N <u_k|x><x'|u_k>=$ \\$=\sum_{k=1} ^N <u_k|x><x|u_k>=\sum_{k=1} ^N |<u_k|x>|^2$ \\Infatti si ha che $<x'|u_k>=<\sum_{l=1} ^N <u_l|x> u_l|u_k>=\sum_{l=1} ^N \overline{<u_l|x>} <u_l|u_k>=$ \\$=\sum_{l=1} ^N <x|u_l> \delta_{lk}=<x|u_k>$. \\Possiamo dimostrarlo anche partendo dalla condizione $<x-x'|x'> =0$; infatti: \\$<x-x'|x'>=<x|x'>-<x'|x'>=\sum_{k=1} ^N <u_k|x><x|u_k>-\|x'\|^2=0$, e da qui si ottiene la disuaglianza di Bessel.
\item \textbf{Disuguaglianza di Schwarz}: $|<x|y>|=\|x\| \|y\|$ \\Siano $x,y,\lambda$ dei numeri complessi; si ha che: \\$0 \leq \|x+\lambda y\|^2=<x+\lambda y|x+\lambda y>=\|x\|^2+|\lambda|^2 \|y\|^2 + \overline{\lambda}<y|x> + \lambda <x|y> \leq$ \footnote{Si usa la proprietà dei numeri complessi per la quale $\overline{x} + x =2  Re(x) \leq 2|x|$.} $\leq \|x\|^2 +|\lambda|^2 \|y\|^2 +2|\lambda||<x|y>|$ $\forall \lambda$ \\Possiamo vedere tale risultato come un'equazione quadratica nella variabile $|\lambda|$; cerchiamo la condizione $\Delta \leq 0$: \\$|<x|y>|^2 - \|x\|^2 \|y\|^2 \leq 0 \implies |<x|y>|^2 \leq \|x\|^2 \|y\|^2$ \\Utilizzando tale relazione per maggiorare il risultato precendente e ponendo $\lambda=1$ otteniamo che: \\$\|x+y\|^2 \leq  \|x\|^2 + \|y\|^2 +2|<x|y>|^2 \leq \|x\|^2 +\|y\|^2+2\|x\|^2 \|y\|^2=(\|x\| + \|y\|)^2$ \\ e quindi $\|x+y\| \leq \|x\|+\|y\|$, cioè la tesi.
\end{itemize} Quindi $\|x\|=\sqrt{<x|x>}$ è una \textbf{norma (hilbertiana)}, poichè valgono le seguenti proprietà:

\begin{itemize}
\item $\|x\| \geq 0$, e $\|x\|=0 \iff x=0$
\item$\| \lambda x\|=| \lambda| \|x\|$
\item Vale una disuguaglianza triangolare, cioè la disuguaglianza di Schwarz
\end{itemize} Le proprietà della norma hibertiana sono:

\begin{itemize}
\item $\|x \pm y\|^2= \|x\|^2 +\|y\|^2 \pm Re(<x|y>)$
\item Vale la \textbf{proprietà del parallelogrammo}, cioè si ha che $\|x+y\|^2+\|x-y\|^2=2\|x\|^2 + 2 \|y\|^2$
\end{itemize}

\begin{teorema}

Condizione necessaria e sufficiente affinchè una norma sia  hilbertiana (cioè che discenda da un prodotto interno) è che valga la proprietà  del parallelogrammo.
\end{teorema}

\begin{proof} Se la norma è hilbertiana, la dimostrazione è ovvia.

Se vale la proprietà del parallelogrammo, si dimostra che $\frac{1}{4} (\|x+y\|^2-\|x-y\|^2)$ e  $\frac{1}{4} (\|x-iy\|^2-\|x+iy\|^2)$ sono rispettivamente la parte reale e la parte immaginaria in un prodotto interno $<x|y>$; a questo punto, verificando le proprietà del prodotto interno, si ricava che quello appena definito è un prodotto interno solo se vale la proprietà del parallelogrammo.

\end{proof}

La formula del prodotto interno definita nella dimostrazione è detta \textbf{formula di polarizzazione} ed è qui riportata integralmente:

$<x|y>= \frac{1}{4} (\|x+y\|^2-\|x-y\|^2)+ \frac{i}{4} (\|x-iy\|^2+\|x+iy\|^2)$ \\
\\
Uno spazio con prodotto interno è detto \textbf{spazio normato}; se è completo (cioè se tutte le successioni di Cauchy sono convergenti), lo spazio è detto \textbf{spazio di Hilbert}.
\\
\\
L'operazione prodotto interno $<x|\, \cdot>:\mathscr{H} \to \C$ è un'applicazione lineare ed è continui; se una successione $y_n$ converge a $y$ in $\mathscr{H}$, allora il prodotto interno $<x|y_n>_{\C}$ converge a $<x|y>$. nCosa vuol dire che la successione converge? Vuol dire che la quantità $|<x|y_n>-<x|y>|$ tende a $0$; infatti possiamo scrivere che $|<x|y_n>-<x|y>|=|<x|y_n-y>|$ e, per la disuguaglianza di Schwarz, si ha che $|<x|y_n-y>| \leq \|x\| \|y_n-y\|$ e, dato che $y_n \to y$, si ha che il tutto tende a $0$. \\Quello che abbiamo appena analizzato è un esempio di \textbf{funzionale lineare continuo}, cioè un'applicazione lineare a valori in $\C$ e continua. \\Quindi il prodotto interno $<x|\, \cdot> \in \mathscr{H}^*$, cioè appartiene all'insieme dei funzionali lineari continui di $\mathscr{H}$ (detto \textbf{duale di $\mathscr{H}$}).

Tutti gli elementi di $\mathscr{H}^*$ sono della forma $<x|\, \cdot>$, cioè gli elementi del duale di $\mathscr{H}$ sono in corrispondenza biunivoca con gli elementi di $\mathscr{H}$.
\\
Due spazi di Hilbert $\mathscr{H}_1$ e $\mathscr{H}_2$ si dicono \textbf{isomorfi} se esiste una mappa $\hat{U}:\mathscr{H}_1 \to \mathscr{H}_2$ lineare invertibile; inoltre, tale $\hat{U}$ conserva la norma, cioè $\|\hat{U} x \|_2=\|x\|_1$ $\forall x \in \mathscr{H}_1$. \\Per effetto della formula di polarizzazione, si ha che siffatta mappa conserva il prodotto interno, cioè $<\hat{U} x|\hat{U}y>_2=<x|y>_1$ $\forall x,y$; tale proprietà è detta \textbf{isomorfismo hilbertiano}.

L'operatore $\hat{U}$ è un \textbf{operatore lineare unitario}. Tutte le simmetrie continue sono rappresentate da operatori unitari.
\\
\\
Esempi:
\begin{itemize}
\item Gli insiemi $\R^n$ e $\C^n$ in cui il prodotto interno è definito come il prodotto scalare, cioè in cui $<u|v>=\sum_i u_i v_i$
\item L'insieme $\C^{n \times n}$, cioè l'insieme delle matrici $n \times n$ a coefficienti complessi, in cui il prodotto interno è definito come $<A|B>=Tr(A^+ B)$; l'operatore ''$^+$'' indica che viene presa la matrice trasposta coniugata, cioè si ha che $(A^+)_{ij}=\overline{A}_{ji}$.
\end{itemize}

Introduciamo ora lo spazio $l^2(\C)$, cioè lo spazio delle successioni numeriche in $\C$ tali che la serie dei quadrati converga; quindi un elemento  $a \in l^2(\C)$ e una successione $\{ a_n\}_{n=0} ^{\infty}$ tale che ogni $a_n \in \C$ e $\sum_i |a_n|^2 < \infty$. Lo spazio $l^2(\C)$ è uno spazio lineare, poichè valgono: 
\begin{itemize}
\item $a,b \in l^2(\C) \implies a+b=\{a_n + b_n\} \in l^2(\C)$
\item $a \in l^2(\C) \implies \lambda a=\{ \lambda a_n\} \in l^2(\C)$
\end{itemize}
Il prodotto interno per lo spazio $l^2(\C)$ è definito come:
$$<a|b> = \sum_{n=0} ^{\infty} \overline{a_n} b_n$$
e quindi la norma quadra di un elemento $a \in l^2(\C)$ è definita come:
$$\sum_{n=0} ^{\infty} |a_n|^2 < \infty$$
Per verificare che quello definito sopra sia un prodotto interno, bisogna verificare che la serie converge $\forall a,b \in l^2(\C)$.

Dimostriamo ora il seguente teorema:
\begin{teorema}
Lo spazio $l^2(\C)$ è completo, cioè ogni successione di Cauchy converge.
\end{teorema}
\begin{proof}
Sia $a_n$ una successione di Cauchy; vale allora che
$$\forall \epsilon >0 \text{ } \exists N_{\epsilon} \text{ : } \|a_n-a_m\|_2 <\epsilon \text{ } \forall n,m>N_{\epsilon}$$
Quindi possiamo scrivere:
$$\|a_n-a_m\|^2= \sum_{k=0} ^{\infty} |a_n ^{(k)} - a_m ^{(k)}|^2 < \epsilon ^2 \implies \forall |a_n ^{(k)} - a_m ^{(k)}|^2 < \epsilon ^2 \text{ } \forall n,m>N_{\epsilon}, \forall k=0,1,2,\dots$$
Allora abbiamo costruito una successione \footnote{La successione è di indice n; l'indice k ci serve per definire la successione dei limiti.}$a_n ^{(k)}$ che è di Cauchy e converge ad un certo $a^{(k)}$ $\forall k$\footnote{Si ha cioè che $a_n ^{(1)} \to a^{(1)}, a_n ^{(2)} \to a^{(2)} \dots$}. Presa la successione di tali limiti, cioè $\{a^{(k)}\}$, sia $a$ il limite di tale successione. Per $m \to \infty$, si ha che:
$$\sum_{k=0} ^{\infty} |a_n ^{(k)} - a_m ^{(k)}|^2= \sum_{k=0} ^{\infty} |a_n ^{(k)} - a^{(k)}|^2< \epsilon ^2$$
Dato che $a_n \in l^2(\C)$ e dato che l'elemento $|a_n -a|$ appartiene a $l^2(\C)$, per linearità si ha che  anche $a \in l^2(\C)$. Poichè $a_n \to a$, si ha la tesi.

\end{proof}

\section{Lo spazio $\mathscr{L}^p (\Omega)$}

Lo spazio  $\mathscr{L}^p (\Omega)$, dove $\Omega$ è un'insieme misurabile, è uno spazio di funzioni per le quali vale che
$$\int dx |f|^p < \infty$$
La norma in tale spazio vale:
$$\|f\| _p=\sqrt[p]{\int dx |f|^p}$$

Introduciamo ora due importanti risultati:
\begin{itemize}
\item \textbf{Disuguaglianza di H\"{o}lder} \\Siano $f \in \mathscr{L}^p (\Omega)$ e $g \in \mathscr{L}^q (\Omega)$, con $\frac{1}{p} + \frac{1}{q} =1$; allora si ha che \\$$\int_{\Omega} |fg|dx \leq \|f\|_p \|g\|_q$$
\item \textbf{Disuguaglianza di Minkowski} \\Siano $f, g \in  \mathscr{L}^p (\Omega)$. Allora si ha che \\$$\|f+g\|_p \leq \|f\|_p + \|g\|_p$$
\end{itemize}
Non dimostreremo queste due disuguaglianze, ma le citiamo perchè da esse possiamo ricavare delle importanti proprietà; infatti, dalla disuguaglianza di Minkowski ricaviamo che:
\begin{itemize}
\item Siano $f,g \in \mathscr{L}^p (\Omega)$; allora si ha che anche $f+g \in \mathscr{L}^p (\Omega)$
\item Sia $f \in \mathscr{L}^p (\Omega)$ e sia $\lambda \in \C$; allora si ha che $\lambda f \in \mathscr{L}^p (\Omega)$
\end{itemize}
Cioè otteniamo che lo spazio $\mathscr{L}^p (\Omega)$ è lineare.
Dovremmo quindi dimostrare le tre proprietà degli spazi lineari:
\begin{itemize}
\item $\| \lambda f\|_p=|\lambda| \|f\|_p$
\item $\|f+g\|_p \leq \|f\|_p + \|g\|_p$
\item $\|f\|_p \geq 0$
\end{itemize}
Prima di tutto però ci chiediamo: vale che $\int dx |f|^p =0 \iff f=0$? Non vale, altrimenti si avrebbe che $f=0$ q.o.\footnote{q.o. è l'abbreviazione di ``quasi ovunque'', che significa che la proprietà citata vale sempre tranne che in insiemi di misura nulla.} (e per gli integrali gli insiemi di misura nulla non contano, cioè il contributo di un insieme di misura nulla ad un integrale è zero). \\Come risolviamo questo problema? Introduciamo una relazione di equivalenza $f \tilde g$ che mette in relazione le funzioni $f$ e $g$ se si ha che $f=g$ q.o.; possiamo quindi dividere le funzioni per classi di equivalenza $[f]$. \\Prese due classi di equivalenza $[f]$ e $[g]$, ognuna di esse avrà un rappresentante (che indicheremo rispettivamente con $f$ e $g$); abbiamo che valgono le proprietà di linearità, cioè si ha che $[f]+[g]=[f+g]$ e $\lambda [f]=[\lambda f ]$.

Tutti gli elementi di una stessa classi di equivalenza hanno la stessa norma a p ($\|$ $\|_p$); oltretutto, quando si ha che $\|f\|_p=0$, allora l'intera classe di equivalenza $[f]$ è l'elemento nullo.

Quindi l'insieme $\mathscr{L}^p (\Omega)$ contiene classi di equivalenza di funzioni misurabili, con $\int_{\Omega} dx |f|^p < \infty$. Esso è uno spazio lineare (per quanto visto prima), ed è anche normato con $\|f\|_p=\sqrt[p]{\int_{\Omega} dx |f|^p}$. Inoltre uno spazio di questo tipo (con $p \geq 1$) è uno spazio completo, cioè $f_n \to f$ in $\mathscr{L}^p (\Omega)$ se $\|f_n-f\|_p \to 0$.
\begin{teorema} (di Fisher-Riegz)\\
$\mathscr{L}^p (\Omega)$ è uno spazio completo (\textbf{spazio di Banach}) se $p \geq 1$.
\end{teorema}
Noi useremo principalmente due di questi spazi: lo spazio $\mathscr{L}^1 (\Omega)$, cioè lo spazio delle funzioni integrabili (infatti si ha che la condizione sulla norma rappresenta la condizione di integrabilità, $\|f\|_1 =\int_{\Omega} dx |f| < \infty$), e lo spazio $\mathscr{L}^2 (\Omega)$; per quest'ultimo, abbiamo la particolarità del fatto che la norma che abbiamo definito è una norma hilbertiana, perchè si ha $\|f\|_2=\sqrt{\int_{\Omega} dx |f|^2} < \infty$. Quindi lo spazio $\mathscr{L}^2 (\Omega)$ è uno spazio di hilbert, con prodotto interno definito come:
$$<f|g>=\int_{\Omega} \overline{f} g dx$$
dove abbiamo preso i rappresentanti delle due classi di equivalenza $[f]$ e $[g]$.

\begin{osservazione} Se la misura dell'insieme $\Omega$ è finita, cioè se si ha che $\mu (\Omega) < \infty$, allora la funzione $f=cost$ $\in \mathscr{L}^p (\Omega)$. Sia infatti $\int_{\Omega} 1 |f| dx$, con $\mu(\Omega) < \infty$, e supponiamo che $f \in \mathscr{L}^2 (\Omega)$; allora si ha che:
$$\|f\|_1=\int_{\Omega} 1 |f| dx=<1|f>_2 \leq \footnote{Per la disuguaglianza di Schwarz} \|1\|_2 \|f\|_2$$
dove $\|1\|_2$ è proprio la misura $\mu(\Omega)$ che è finita; quindi se $f \in \mathscr{L}^2 (\Omega)$ ai ha anche che $f \in \mathscr{L}^1 (\Omega)$.
\end{osservazione}

\section{Sistemi ortogonali}

Sia $\mathscr{H}$ uno spazio di Hilbert qualunque, e sia $\mathcal{M}$ un suo sottoinsieme. Esso è un sottospazio se è chiuso per addizione e moltoplicazione per uno scalare in $\C$.








